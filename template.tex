%%-+- coding:UTF-8  -+-
%%fyh的常用模板
%%历史记录
%%2019/04/11         fyh          建立模板并边学习


%预设置---------------------------------------------------%
%-+- coding:UTF-8  -+-
%fyh的常用模板
%readme-----------------------------------------------------%
%包含以下部分的定义:
%0、定义文章类别为报告形式
%1、引用宏包:
%2、定义页面:纸张大小、页眉页脚等
%3、字体字号设置-标题、正文等
%4、图片、公式相关设置
%5、引用参考文献设置
%readme-----------------------------------------------------%


%0、定义文章类型
\documentclass[UTF8,a4paper]{ctexrep}  


%1、引用宏包---------------------------------------------------%
\usepackage{ctex}   %中文宏包
\usepackage{geometry}  %设置页面大小
\usepackage[hidelinks]{hyperref}   
\usepackage{amsmath}	%数学公式编号宏包
\usepackage{cite}        %引用参考文献宏包
\usepackage{url}		 %网页引用链接
\usepackage{graphicx}    %图片宏包
\usepackage{subfigure}   %子图宏包
\usepackage{caption}	%注释宏包
\usepackage{listings}    %插入代码包
\usepackage{xcolor}
\usepackage{fontspec} 
\usepackage{xeCJK}		%中文字体
\usepackage{setspace}
\usepackage{fancyhdr}	%页眉页脚
\usepackage{lastpage}
%引用宏包---------------------------------------------------%


%2、定义页面:纸张大小、页眉页脚等------------------------------------------------%
%纸张规格为A4 ;版面上空2.5cm,下空2cm,左空2.5 cm,右空2 cm
\geometry{left=2.5cm,right=2cm,top=2.5cm,bottom=2cm}

%设置页眉页脚
%\footskip = 20pt                                                
\pagestyle{fancy}
\fancypagestyle{plain}{
	\pagestyle{fancy}
}
\fancyhead{} % clear all header fields
\fancyfoot{} % clear all footer fields

\fancyhead[L]{{\leftmark}}
\fancyhead[C]{\small 华北电力大学XXXX}
\fancyhead[R]{{\rightmark}}
%%设置页眉  
\fancyfoot[L]{\textsl{页脚左}}
\fancyfoot[C]{\thepage$\backslash$\pageref{LastPage}}
\fancyfoot[R]{\textsl{页脚右}}

\renewcommand{\headrulewidth}{0.4pt}
\renewcommand{\footrulewidth}{0.4pt}    
          
%%设置页眉页脚

%定义页面:纸张大小、页眉页脚等------------------------------------------------%


%3、字体字号设置-标题、正文等--------------------------------------------%
% 正文中文字体字号:宋体小四  正文英文字体字号:新罗马字体 
%\newcommand{\textch}{\setmainfont{SimSun}\zihao{-4}}
%\setmainfont{隶书}
\setCJKmainfont{SimSun} 
\newcommand{\textsize}{\zihao{-4}}
\setmainfont{Times New Roman}

%添加关键词关键词三个字左顶格,用宋体小四号加粗,3-5个关键词用宋体小四号,
%词于词之间用逗号隔开,最后一个词不加任何标点符号。
%关键词左顶格,Times New Roman字体、小四加粗,3-5个关键词Times New Roman字体、小四号,

% 设置标题字体样式
\ctexset{
	chapter = {
		format+ = \setmainfont{SimHei}\zihao{1},
	},
}
%字体字号设置-标题、正文等--------------------------------------------%  


%4、图片、公式相关设置------------------------------------------%
\numberwithin{figure}{section}  %分章节显示图片
\newcommand{\upcite}[1]{\textsuperscript{\textsuperscript{\cite{#1}}}}   %分章节显示公式
%图片、公式相关设置------------------------------------------%


%5、参考文献设置---------------------------------------------------%
\bibliographystyle{plain}

%参考文献设置---------------------------------------------------%
%预设置---------------------------------------------------%

	


%正文区---------------------------------------------------%
\begin{document}
	%正文字体大小设置
	\textsize
	
	%封面---------------------------------------------------
	
\newcommand \dunderline[3][-1pt]{{%
		\setbox0=\hbox{#3}
		\ooalign{\copy0\cr\rule[\dimexpr#1-#2\relax]{\wd0}{#2}}}}
	
\begin{titlepage}
  \vspace*{2cm}
  \centering
  		
	\includegraphics[scale = 0.5]{fig/NCEPU_logo.png}
  
  \vspace{1cm}

  \zihao{1}\bf{{本科生毕业论文(设计)}}
  
  \vspace{14mm}

  \zihao{1}\bf{ 我是大标题XXXX}

  \vspace{3mm}

%  \begin{spacing}{1.2}
%    \LARGE\selectfont{\textbf{\heiti ——我是小标题}}
%  \end{spacing}

  \vspace{10mm}

\begin{flushleft}
	  \begin{spacing}{1.3}
		\hspace{27mm}\LARGE\selectfont{\textbf{论文编码:}\dunderline[-3pt]{1pt}{\makebox[78mm][c]{XXXXXXXX}}}
		
		\hspace{27mm}\LARGE\selectfont{\textbf{学\hspace{12.5mm}院:}\dunderline[-3pt]{1pt}{\makebox[78mm][c]{电气与电子工程学院}}}
		
		\hspace{27mm}\LARGE\selectfont{\textbf{专\hspace{12.5mm}业:}\dunderline[-3pt]{1pt}{\makebox[78mm][c]{电气工程及其自动化}}}
		
		\hspace{27mm}\LARGE\selectfont{\textbf{年\hspace{12.5mm}级:}\dunderline[-3pt]{1pt}{\makebox[78mm][c]{2017级}}}
		
		\hspace{27mm}\LARGE\selectfont{\textbf{学\hspace{12.5mm}号:}\dunderline[-3pt]{1pt}{\makebox[78mm][c]{120171030408}}}
		
		\hspace{27mm}\LARGE\selectfont{\textbf{学生姓名:}\dunderline[-3pt]{1pt}{\makebox[78mm][c]{付裕恒}}}
		
		\hspace{27mm}\LARGE\selectfont{\textbf{指导教师:}\dunderline[-3pt]{1pt}{\makebox[78mm][c]{XX}}}
		
		\hspace{27mm}\LARGE\selectfont{\textbf{完成日期:}\dunderline[-3pt]{1pt}{\makebox[78mm][c]{XXXXX}}}
	\end{spacing}
\end{flushleft}


  \vspace{25mm}

\end{titlepage}
	\clearpage
	%封面---------------------------------------------------
	
	%目录-----------------------------------------------------
	\clearpage
	\tableofcontents	
	\clearpage
	%目录-----------------------------------------------------
	
	%正文-----------------------------------------------------------
	%摘要-------------------------------------------------------
		\chapter*{摘\qquad 要}
		
		这是一个摘要示例:
		
		感谢老师这学期把我引入了这一个奇妙的领域。以前总害怕直接去看大段英文的东西,但是硬着头皮看下来以后感觉体会很多,虽然书中众多的公式令我眼花缭乱,但是我也了解到这个领域需要很多的数学知识,例如:高等代数、有限差分法、变分法等。明确了我下一阶段要学习的基础知识。通过观看学习oommf官方教程,一是体会到仿真对于理解问题会很有帮助,二是体会到要进一步去学习C++等编程相关的知识,如果未来开发,可能会很有帮助。
		
		\begin{flushleft}
			\textbf{关键词:}心得,C++    %实际未加粗,待解决
		\end{flushleft}
		
		\chapter*{Abstract}
		
		This is a summary example:
		
		Thanks to the teacher for introducing me into this wonderful field this semester. I used to be afraid to look directly at large sections of English, but after I bite the bullet and read it, I feel a lot of experience. Although the many formulas in the book dazzle me, I also understand that this field requires a lot of mathematical knowledge, such as advanced algebra. 
		, Finite difference method, variational method, etc. 
		I clarified the basic knowledge that I will learn in the next stage. By watching the official oommf tutorial, one is to realize that simulation is very helpful for understanding the problem, and the other is to realize that you need to further learn C++ and other programming-related knowledge. If you develop in the future, it may be very helpful.
		
		\begin{flushleft}
			\textbf{KEYWORDS:}experience,C++
		\end{flushleft}
	%摘要-------------------------------------------------------


	\chapter{磁化动力学基础}
	
	\section{磁化动力学基本方程}
	
	\section{磁化动力学方程的归一化形式}
	
	\section{磁化强度的翻转过程}
	
	\chapter{非线性磁化动力学数值积分}
	\section{数学基础:中点法则数值方法}
	\section{中点差分法离散形式的LLG方程}
	
	\chapter{仿真分析}
	\section{OOMMF简介}
	\section{问题描述}
	\section{结果展示与分析}
	%正文-----------------------------------------------------------

	
%\newpage
%%%/***************************************************************/
%%%目录
%%%%%%给第一级标题加点
%\renewcommand{\cftdot}{$\cdot$}
%\renewcommand{\cftdotsep}{1.5}
%\setlength{\cftbeforechapskip}{10pt}
%
%\renewcommand{\cftchapleader}{\cftdotfill{\cftchapdotsep}}
%\renewcommand{\cftchapdotsep}{\cftdotsep}
%\makeatletter
%\renewcommand{\numberline}[1]{%
%\settowidth\@tempdimb{#1\hspace{0.5em}}%
%\ifdim\@tempdima<\@tempdimb%
%  \@tempdima=\@tempdimb%
%\fi%
%\hb@xt@\@tempdima{\@cftbsnum #1\@cftasnum\hfil}\@cftasnumb}
%\makeatother
%%%%%%给第一级标题加点
%%黑体四号,目录两个字之间空4个字
%\pagenumbering{arabic}
%\renewcommand\contentsname{目\qquad 录}
%\tableofcontents
%%\tableofcontents{section}{\songti\zihao{-4}}
%%\tableofcontents{subsection}{\songti\zihao{-4}}
%%%/***************************************************************/
%%%图目录
%\renewcommand*{\listfigurename}{图目录}
%\listoffigures
%\addcontentsline{toc}{chapter}{图目录}
%%%/***************************************************************/
%%%表目录
%\renewcommand*{\listtablename}{表目录}
%\listoftables
%\addcontentsline{toc}{chapter}{表目录}
%
%%黑体小四号:摘要、ABSTRACT、一级标题
%%宋体小四号:二级三级
%\clearpage
%\pagenumbering{arabic}
%
%%%/***************************************************************/
%%%正文
%\zihao{-4}\songti
%%%/***************************************************************/
%%%第一章
%\chapter{绪论}{\heiti\zihao{-2}}
%\thispagestyle{fancy}
%
%  \section{课题背景}
%  这里引用文献\cite{REF_13浅探应急通信保障中无线自组网技术的应用}。
%  \section{国内外研究现状}
%    \subsection{电能质量监测国内外研究现状}
%    \subsection{无线自组网的国内外研究现状}
%    \thispagestyle{fancy}
%
%  \section{本文研究内容和结构组织}
%
%%%/***************************************************************/
%%%/***************************************************************/
%%%/***************************************************************/
%%%第二章
%\clearpage
%\chapter{总体设计}{\heiti\zihao{-2}}
%\thispagestyle{fancy}
%\section{相关技术介绍}
%  \subsection{STM32嵌入式程序设计}
%  这里插入图片。
%  %%图片2-1插入
%
%  \begin{figure}[ht]
%  \centering
%%  \includegraphics[width=15cm]{Picture/图2-1STM32L4系列.png}
%  \caption{STM32L4系列}
%  \label{fig:STM32L4系列}
%  \end{figure}
%
%
%  %此处放软件编程
%  %%/***************************************************************/
%  \subsection{无线自组网}
%  %%图片2-2插入
%  表格插入
%
%  \begin{table}[ht]
%    \centering
%    \caption{网路设备各层及所需确定的参数表}
%    \begin{tabular}{c|c}
%      \toprule
%      网络设备  & 配置参数 \\
%      \hline
%      MAC     &   MAC协议选择及参数\\
%      \hline
%      信道    &    传输损耗、延时等参数\\
%      \hline
%      物理层    &   无线网卡(硬件设备及驱动)、工作频率、发射频率、接收门限、噪声等\\
%      \bottomrule
%    \end{tabular}
%  \end{table}
%  
%\begin{figure}[ht]
%  \centering
%%  \subfigure[蜂窝移动通信拓扑示意图]
%%  {
%%    \begin{minipage}{6cm}
%%    \centering
%%    \includegraphics[width=6cm]{Picture/图2-2-1蜂窝移动通信拓扑示意图.pdf}
%%    \end{minipage}
%%  }
%%  \subfigure[无线自组网网络拓扑结构图]
%%  {
%%    \begin{minipage}{6cm}
%%      \centering
%%      \includegraphics[width=6cm]{Picture/图2-2-2无线自组网网络拓扑结构图.pdf}
%%      \end{minipage}
%%  }
%  \caption{蜂窝移动通信拓扑示意图和无线自组织网络网络拓扑示意图}
%  \label{fig:蜂窝移动通信拓扑示意图和无线自组织网络网络拓扑示意图}
%\end{figure}
%  %%/***************************************************************/
%  \subsection{神经网络}
%  %%/***************************************************************/
%  \subsection{电能质量分析}
%%%/***************************************************************/
%%%/***************************************************************/
%\section{系统总体架构设计}
%
%%%/***************************************************************/
%%%第三章
%\chapter{仿真结果与分析}{\heiti\zihao{-2}}
%\thispagestyle{fancy}
%\section{ns-3网络模拟器与仿真结果与分析}
%\section{基于神经网络的电能质量信息处理结果与分析}
%%%/***************************************************************/
%%%第四章
%\chapter{总结与展望}{\heiti\zihao{-2}}
%\thispagestyle{fancy}
%\section{总结}
%\section{展望}
%\clearpage
%%重新用阿拉伯文字记页码数
%%%/***************************************************************/
%%%参考文献
%\pagenumbering{arabic}
%%\chapter*{参考文献}{\heiti\zihao{-2}}
%\bibliography{参考文献}
%\addcontentsline{toc}{chapter}{参考文献}
%\thispagestyle{fancy}
%%%/***************************************************************/
%%%附录
%\chapter*{附\qquad 录}{\heiti\zihao{-2}}
%\addcontentsline{toc}{chapter}{附\qquad 录}
%\thispagestyle{fancy}
%论文的附录依次按附录A,附录B 等进行编号。附录内容的书写格式按毕业设计(论文)的正文规定格式书写。
%%%/***************************************************************/
%%%致谢
%\chapter*{致\qquad 谢}{\heiti\zihao{-2}}
%\addcontentsline{toc}{chapter}{致\qquad 谢}
%\thispagestyle{fancy}
%对曾经给予本人顺利完成毕业设计(论文)而提供各类帮助、指导,以及协助完成该项研究工作的单位和个人表示感谢。
%

\end{document}
%正文区---------------------------------------------------%
