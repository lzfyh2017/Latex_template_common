%-+- coding:UTF-8  -+-
%fyh的常用模板
%readme-----------------------------------------------------%
%包含以下部分的定义:
%0、定义文章类别为报告形式
%1、引用宏包:
%2、定义页面:纸张大小、页眉页脚等
%3、字体字号设置-标题、正文等
%4、图片、公式相关设置
%5、引用参考文献设置
%readme-----------------------------------------------------%


%0、定义文章类型
\documentclass[UTF8,a4paper]{ctexrep}  


%1、引用宏包---------------------------------------------------%
\usepackage{ctex}   %中文宏包
\usepackage{geometry}  %设置页面大小
\usepackage[hidelinks]{hyperref}   
\usepackage{amsmath}	%数学公式编号宏包
\usepackage{cite}        %引用宏包
\usepackage{url}		 %网页引用链接
\usepackage{graphicx}    %图片宏包
\usepackage{subfigure}   %子图宏包
\usepackage{caption}	%注释宏包
\usepackage{listings}    %插入代码包
\usepackage{xcolor}
\usepackage{fontspec} 
\usepackage{xeCJK}		%中文字体
\usepackage{setspace}
%引用宏包---------------------------------------------------%


%2、定义页面:纸张大小、页眉页脚等------------------------------------------------%
%纸张规格为A4 ;版面上空2.5cm,下空2cm,左空2.5 cm,右空2 cm
\geometry{left=2.5cm,right=2cm,top=2.5cm,bottom=2cm}

%%设置页眉页脚
%\fancypagestyle{plain}{%
%
%\fancyhf{} % 清空当前设置
%
%%设置页眉 (head)
%\fancyhead[L,R]{}
%\fancyhead[C]{华北电力大学xxxxx}
%\pagestyle{fancy}
%%设置页眉页脚
%定义页面:纸张大小、页眉页脚等------------------------------------------------%


%3、字体字号设置-标题、正文等--------------------------------------------%
% 正文中文字体字号:宋体小四  正文英文字体字号:新罗马字体 
%\newcommand{\textch}{\setmainfont{SimSun}\zihao{-4}}
\setCJKmainfont{SimSun} 
\newcommand{\textsize}{\zihao{-4}}
\setmainfont{Times New Roman}

%添加关键词关键词三个字左顶格,用宋体小四号加粗,3-5个关键词用宋体小四号,
%词于词之间用逗号隔开,最后一个词不加任何标点符号。
%关键词左顶格,Times New Roman字体、小四加粗,3-5个关键词Times New Roman字体、小四号,

% 设置标题字体样式
\ctexset{
	chapter = {
		format+ = \setmainfont{SimHei}\zihao{1},
	},
}
%字体字号设置-标题、正文等--------------------------------------------%  


%4、图片、公式相关设置------------------------------------------%
\numberwithin{figure}{section}  %分章节显示图片
\newcommand{\upcite}[1]{\textsuperscript{\textsuperscript{\cite{#1}}}}   %分章节显示公式
%图片、公式相关设置------------------------------------------%


%5、参考文献设置---------------------------------------------------%
\bibliographystyle{plain}
%参考文献设置---------------------------------------------------%


%\title{磁化动力学基础及仿真分析}
%\author{付裕恒}
%\date{\today}